\documentclass[11pt]{article}
\usepackage{graphicx}
\usepackage{amssymb}


\begin{document}

Dear Reviewer, dear Editor,

\bigskip

we would like to thank the reviewer for her/his careful reading and thoughtful comments on the submitted paper. 

Below we address each of the comments in details.

\begin{enumerate}
	\item \textbf{``The word ``locality" used in the title is not explained in the paper. The introduction mentions the technique of iterative rounding, but the word ``locality" should explicitly appear to justify its presence in the title, as some readers of this paper may not be familiar with iterative rounding and thus may not understand the meaning of ``locality" in this context."}
	
	\smallskip
	
	We agree with the reviewer that the role of locality was not explained in a clear way. Indeed, not every reader might be comfortable with iterative rounding and hence a detailed explanation of ``locality" might be required. Hence, we decided not to mention ``locality" in order to keep the exposition as simple as possible.
	
	\bigskip
	
	\item  \textbf{``Top of page 2: $M_1\triangle M_2$ is an edge set and not a graph, so what do you mean exactly by ``connected components" of $M_1\triangle M_2$? Are you taking as vertices only the endpoints
of the edges in $M_1\triangle M_2$, or all the vertices of the original graph? (In the latter case,
some connected components are isolated nodes: see point 8 below.)"}
	
	\smallskip
	
	We added a disclaimer: ``Note that $M_1\triangle M_2$ is an edge set, however in this paper we refer to nontrivial connected components of the graph $(V, M_1\triangle M_2)$ as connected components of $M_1\triangle M_2$.".
	
	\bigskip

	\item  \textbf{``Page 2, line 13: it might be useful to say that the edges of $C$ alternates between $M_1$
and $M_2$."}
	
	\smallskip
	
	Done.
	
	\bigskip
	
	\item \textbf{``Page 2, line 15: Since $v$ has been already fixed, while $a$ is being defined here, I think
one should write $v =: a_2 \in \mathcal{A}$ instead of $v := a_2\in \mathcal{A}$."}
	
		
	\smallskip
	
	Done.
	
	\bigskip
	
	\item \textbf{``Figure 1: $a_1$ and $b_1$ do not match the notation used in the proof. Please either make
the correspondence explicit in the caption (or in the proof), or use the same notation."}
	
	\smallskip
	
	Done.
	
	\bigskip
	
	\item \textbf{``Lemma 2.2: do you mean that $J_1$ and $J_2$ are the edge sets of the connected components? Otherwise notation $M_1\triangle J_1$ does not make sense."}
	
	\smallskip
	
	Thanks. We changed the notation from $M_1\triangle J_1$ to $M_1\triangle E(J_1)$. Please, see also Comment 2.
	
	\bigskip
	
	\item \textbf{``Page 2, line 39: does this follow from Lemma 2.1? If so, please mention this."}
	
	\smallskip
	
	It is due to the fact that if $a$ and $b$ both lie in $J_1$ then the neighbours of $a$ and $b$ in $M_1\triangle E(J_1)$ are the respective neighbours of $a$ and $b$ in $M_2$. Thus the edge $ab$ is also blocking for $M_2$, contradicting stability of $M_2$.
	
	\bigskip
	
		
	\item \textbf{``Page 2, line 40. This is related to point 2 above. If isolated nodes are not counted as
connected components of $M_1\triangle M_2$, then the case of $a$ or $b$ being an isolated node of
$M_1\triangle M_2$ is not covered by the cases ``$a \in V (J_1)$; $b \in V (J_2)$" and ``$a \in V (J_2)$; $b \in V (J_1)$". Otherwise, if isolated nodes are taken into account, then the definition of $J_1$ and $J_2$ does
not clarify whether isolated nodes are components belonging to $J_1$ or $J_2$."}
	
	\smallskip
	
	Thank you for noticing that isolated nodes are not covered by the case distinction we made. Neither $a$ or $b$ can be an isolated node due to the reasoning similar to the answer to comment 7. We added a corresponding explanation.
	
	\bigskip

	\item \textbf{``Figure 2: same observation as in point 5 above. Furthermore, to help the reader I
suggest to illustrate the first case of the proof instead of the second."}
	
	\smallskip
	
       We changed both the caption of the picture as well as the picture itself to illustrate the first case of the proof.
	
	\bigskip

	\item \textbf{``10. Page 3, line 54/55: $\chi$ should be replaced with $x$ (twice)?"}
	
	\smallskip
	
	We changed the formula and explanation here to not use $\chi$.
	
	\bigskip
	
	\item \textbf{``11. Page 3, line 55/56: if I understand correctly, $a$ should be assumed to be in $V_x$."}
	
	\smallskip
	
	We do not need this assumption.
	
	\bigskip
	
	\item \textbf{``Same line: please insert a comma after ``$\varnothing$"."}
	
	\smallskip
	
	Done.
	
	\bigskip
	
	\item \textbf{``Page 3, last line: though its meaning is clear, notation $\delta^{\ge b}(a)$ is not defined in the paper (only $\delta^{> b}(a)$ is defined). The same applies to $\delta^{< b}(a)$ and $N_{\min}(b)$ (only $N_{\max}(b)$ is defined)."}
	
	\smallskip
	
	In the first draft of the paper, we have defined all this notation. However, since the notation is self-explanatory we decided not to introduce all the notation above.
	
	\bigskip
	
	\item \textbf{``Page 4, first line: are you assuming that $e\in E_x$? I think you are making a proof by
contradiction and thus implicitly assuming $e\in E_x$, but this is not stated explicitly."}
	
	\smallskip
	
	We changed the wording here.
	
	\bigskip
	
	
	\item \textbf{``Page 4, line 8: I do not understand where the term ``$-1$" comes from."}
	
	\smallskip
	
	We changed the wording above, to make it clear that for every $b$ the edge $b N_{\max}(b)$ does not lie in $E_x$.
	
	\bigskip
	
	\item \textbf{``Page 4, line 18: please explain why."}
	
	\smallskip
	
	Done.
	
	\bigskip
	
		\item \textbf{``Page 4, line 19: $P(G')$ is an integral polytope by definition 2.3! Do you mean $Q(G')$? But in this case I cannot follow the rest of the proof, as you use Corollary 2.5, which applies really to $P$ and not $Q$. . ."}
	
	\smallskip
	
	By minimality of $G$, we have $P(G')=Q(G')$. We changed the wording here.
	
	\bigskip
	
	\item \textbf{``Last line before references: I think $P(G')$ should read $Q(G')$."}
	
	Please see Comment 17.

	
\end{enumerate}

 Sincerely,

\smallskip

\qquad Jochen K\"{o}nemann

\qquad  Kanstantsin Pashkovich

\qquad  Justin Toth

\end{document}