\documentclass[preprint]{elsarticle}
%\usepackage[margin=1in, bottom=1in, top=1in]{geometry} %1 inch margins
\usepackage{amsmath, amssymb, amstext}
\usepackage{fancyhdr}
\usepackage{algorithm}
\usepackage{algpseudocode}
\usepackage{mathtools}
\usepackage{theorem}
\usepackage{xcolor}

\DeclarePairedDelimiter{\ceil}{\lceil}{\rceil}
\DeclarePairedDelimiter\floor{\lfloor}{\rfloor}

%Theorem
\newtheorem{fact}{Fact}[section]
\newtheorem{lemma}[fact]{Lemma}
\newtheorem{theorem}[fact]{Theorem}
\newtheorem{definition}[fact]{Definition}
\newtheorem{corollary}[fact]{Corollary}
\newtheorem{proposition}[fact]{Proposition}
\newtheorem{claim}[fact]{Claim}
\newtheorem{exercise}[fact]{Exercise}
\newtheorem{note}[fact]{Note}
\newenvironment{proof}{{\bf Proof:  }}{\hfill\rule{2mm}{2mm}}
%Macros
\newcommand{\A}{\mathbb{A}} \newcommand{\C}{\mathbb{C}}
\newcommand{\D}{\mathbb{D}} \newcommand{\F}{\mathbb{F}}
\newcommand{\N}{\mathbb{N}} \newcommand{\R}{\mathbb{R}}
\newcommand{\T}{\mathbb{T}} \newcommand{\Z}{\mathbb{Z}}
\newcommand{\Q}{\mathbb{Q}}
 
 
\newcommand{\cA}{\mathcal{A}} \newcommand{\cB}{\mathcal{B}}
\newcommand{\cC}{\mathcal{C}} \newcommand{\cD}{\mathcal{D}}
\newcommand{\cE}{\mathcal{E}} \newcommand{\cF}{\mathcal{F}}
\newcommand{\cG}{\mathcal{G}} \newcommand{\cH}{\mathcal{H}}
\newcommand{\cI}{\mathcal{I}} \newcommand{\cJ}{\mathcal{J}}
\newcommand{\cK}{\mathcal{K}} \newcommand{\cL}{\mathcal{L}}
\newcommand{\cM}{\mathcal{M}} \newcommand{\cN}{\mathcal{N}}
\newcommand{\cO}{\mathcal{O}} \newcommand{\cP}{\mathcal{P}}
\newcommand{\cQ}{\mathcal{Q}} \newcommand{\cR}{\mathcal{R}}
\newcommand{\cS}{\mathcal{S}} \newcommand{\cT}{\mathcal{T}}
\newcommand{\cU}{\mathcal{U}} \newcommand{\cV}{\mathcal{V}}
\newcommand{\cW}{\mathcal{W}} \newcommand{\cX}{\mathcal{X}}
\newcommand{\cY}{\mathcal{Y}} \newcommand{\cZ}{\mathcal{Z}}


\DeclareMathOperator{\convOp}{conv}
\newcommand{\conv}{\convOp}


\begin{document}
\begin{frontmatter}
%% Title, authors and addresses

%% use the tnoteref command within \title for footnotes;
%% use the tnotetext command for the associated footnote;
%% use the fnref command within \author or \address for footnotes;
%% use the fntext command for the associated footnote;
%% use the corref command within \author for corresponding author footnotes;
%% use the cortext command for the associated footnote;
%% use the ead command for the email address,
%% and the form \ead[url] for the home page:
%%
%% \title{Title\tnoteref{label1}}
%% \tnotetext[label1]{}
%% \author{Name\corref{cor1}\fnref{label2}}
%% \ead{email address}
%% \ead[url]{home page}
%% \fntext[label2]{}
%% \cortext[cor1]{}
%% \address{Address\fnref{label3}}
%% \fntext[label3]{}

%% use optional labels to link authors explicitly to addresses:
%% \author[label1,label2]{<author name>}
%% \address[label1]{<address>}
%% \address[label2]{<address>}

\title{Linear Description for Stable Matching Polytope: Alternative Proof}
\author[co]{Jochen K\"{o}nemann}
\author[co]{Kanstantsin Pashkovich}
\author[co]{Justin Toth\corref{cor1}}
\ead{wjtoth@uwaterloo.ca}
\address[co]{Department of Combinatorics and Optimization, University of Waterloo, Canada}

\cortext[cor1]{Corresponding author}

\begin{abstract}
In this paper we provide an alternative proof for the description.
\end{abstract}
\begin{keyword}
Stable Matching\sep Polytope\sep Extreme Points
\end{keyword}
\end{frontmatter}
{\color{red}
\section*{TODO}
\begin{enumerate}
	\item itroduction
	\item formulate the idea of the proofs by Rothblum and van Vate
	\item citation for structure of symmetric difference, adjacencies in the stable matching polytope
	\item pictures
\end{enumerate}
}
\section{Stable Matching Polytope}

Let $G = (V, E)$ be a bipartite graph  where the bipartition is given by node sets $\mathcal{A}$ and $\mathcal{B}$. Moreover, each node $a\in\mathcal{A}$ defines a total ordering $>_a$ on its neighbour nodes $N(a):=\{b \in\mathcal{B}: \{a,b\}\in E\}$;  the same holds for nodes in $\mathcal{B}$.

Let us introduce the following notation. For $a\in \mathcal{A}$ and $b \in N(\mathcal{A})$, we can define $\delta^{>b}(a):=\{ \{a,\beta\}\in E: \beta>_a b \}$. Similarly we can define $\delta^{>a}(b)$, $\delta^{\leq b}(a)$, $\delta^{\geq b}(a)$, etc. For $a\in \mathcal{A}$ let $N_{\max}(a)$ denote the maximum element in $N(a)$ with respect to $>_a$, analogously define $N_{\min}(a)$ and $N_{\max}(b)$, $N_{\min}(b)$ for $b\in\mathcal{B}$. 

Given a matching $M$ and $v\in V$, we define $M(v)$ as a neighbour of $v$ in $M$ if $v$ is matched by $M$, otherwise $M(v):=\varnothing$. To simplify notation we understand $>_v$, $v\in V$ as a total ordering on the set $N(v)\cup\{\varnothing\}$, where $\varnothing$ is the smallest element with respect to $>_v$.
A matching $M$ in $G$ is called a $\emph{stable matching}$ if for every edge $e=\{a,b\}\in E$ we have
\begin{equation}\label{eq:stability_def}
		M \, \cap \, \big(\delta^{>a}(b) \cup \delta^{>b}(a) \cup  \{e\} \big)\neq\varnothing\,,
\end{equation}
i.e. no edge $\{a,b\}\in E$ is absent in $M$  if $a>_b M(b)$ and $b>_a M(a)$.


Next lemmas study connected components of $M_1\triangle M_2:=\{ e\in E: |M_1\cap \{e\}|+|M_2\cap \{e\}|=1\}$ for stable matchings $M_1$, $M_2$ in $G$. Here, $V(C)$ and $E(C)$ denote the node and edge sets of a graph $C$.

\begin{lemma}\label{lemma:pref}
Let $M_1$ and $M_2$ be two stable matchings in $G$ and let $C$ be a connected component in $M_1 \triangle M_2$. Then, either
\begin{equation}\label{eq:pref_first}
	M_1(a)>_a M_2(a)\,,\,\forall a\in V(C) \cap\mathcal{A}\quad \text{and} \quad  M_1(b)<_b M_2(b)\,,\,  \forall b\in V(C) \cap\mathcal{B}
\end{equation}
or 
\begin{equation}\label{eq:pref_second}
	M_2(a)>_a M_1(a)\,,\,\forall a\in V(C) \cap\mathcal{A}\quad \text{and} \quad  M_2(b)<_b M_1(b)\,,\,  \forall b\in V(C) \cap\mathcal{B}
\end{equation}
\end{lemma}


\begin{proof}
Since $M_1$ and $M_2$ are matchings, $C$ is a path or a cycle. Let $v\in V$ be an end node of $C$ if $C$ is a path, otherwise let $v$ be an arbitrary node of $C$. 

W.l.o.g. $v:=a\in \mathcal{A}$, $M_1(a) >_a M_2(a)$ and $b:=M_1(a)\in\mathcal{B}$. If $a=M_1(b) >_b M_2(b)$, the matching $M_2$ violates~\eqref{eq:stability_def} for the edge $\{a,b\}\in E$. Thus, $a=M_1(b) <_b M_2(b)$. Thus, $M_2(b)\neq \varnothing$ and the matching $M_1$ satisfies~\eqref{eq:stability_def} for the edge $e:=\{b,M_2(b)\}\in E$ only if $M_1(M_2(b))>_{M_2(b)} b$. Continuing in this way, we obtain statement~\eqref{eq:pref_first}.
\end{proof}


\begin{lemma}\label{lemma:sym_stable} 
Let $M_1$ and $M_2$ be two stable matchings in $G$. Let $J_1$ be the union of $E(C)$ for connected components $C$ in $M_1 \triangle M_2$ satisfying~\eqref{eq:pref_first}, let $J_2:=(M_1\triangle M_2)\setminus C$. Then both $M_1\triangle J_1$ and $M_1\triangle J_2$ are stable matchings in $G$.
\end{lemma}
\begin{proof}
By Lemma~\ref{lemma:pref}, $J_2$ is the union of $E(C)$ for connected components $C$ in $M_1 \triangle M_2$ satisfying~\eqref{eq:pref_second}.

Clearly $M'_1:=M_1\triangle J_1$ and $M'_2:=M_1\triangle J_2$ are matchings in $G$. Let us assume that one of these matchings os not stable, w.l.o.g. assume that $M'_1$ does not satisfy~\eqref{eq:stability_def} for some edge $e:=\{a,b\}\in E$ with $a\in\mathcal{A}$ and $b\in\mathcal{B}$. 

Since $M_1$ and $M_2$ are stable, we have $|\{a,b\}\cap V(J_1)|=1$ and $|\{a,b\}\cap V(J_2)|=1$.
If $a\in V(J_1)$ and $b\in V(J_2)$, then $M'_1(a)=M_2(a)$ and $M'_1(b)>_b M_2(b)$, and hence $M_2$ violates~\eqref{eq:stability_def} for $e:=\{a,b\}$. If $a\in V(J_2)$ and $b\in V(J_1)$, then $M'_1(a)=M_1(a)$ and $M'_1(b)>_b M_1(b)$, and hence $M_1$ violates~\eqref{eq:stability_def} for $e:=\{a,b\}$, contradiction.
\end{proof}

\section{Stable Matching Polytope}

Let us define the \emph{stable matching polytope} $P(G)\subseteq\R^E$ for graph $G$ as follows
$$
	P(G):=\conv\{\chi(M)\in\R^E: M \text{ is a stable matching in } G\}\,.
$$
By~\cite{}, $P(G)$ is a nonempty polytope because every graph $G$ has a stable matching.

Clearly, the vertices of $P(G)$ are in one to one correspondence with stable matchings in $G$. Moreover, Lemma~\ref{lemma:sym_stable} helps to understand what pairs of stable matchings in $G$ do not correspond to edges of $P(G)$.

\begin{lemma}\label{lemma:edge}
Let $M_1$ and $M_2$ be two stable matchings in $G$ which define an edge of the polytope~$P(G)$. Then all connected components in $M_1\triangle M_2$ satisfy~\eqref{eq:pref_first} or all of them satisfy~\eqref{eq:pref_second}.  
\end{lemma}
\begin{proof}
If the statement of the lemma does not hold, then in Lemma~\ref{lemma:sym_stable} the sets $J_1$ and $J_2$ are both nonempty, and hence we obtain two stable matchings $M_1\triangle J_1$ and $M_1\triangle J_2$ different from $M_1$, $M_2$ such that $\frac{1}{2}\chi(M_1\triangle J_1)+\frac{1}{2}\chi(M_1\triangle J_2)=\frac{1}{2}\chi(M_1)+\frac{1}{2}\chi(M_2)$, what condradicts the statement that $M_1$ and $M_2$ define an edge of $P(G)$.
\end{proof}


\begin{corollary}\label{cor:edge}
Let $M_1$ and $M_2$ be two stable matchings in $G$ such that
$$
	M_1\cap\delta^{>a}(b)\neq\varnothing, M_1\cap\delta^{>b}(a)\neq\varnothing\quad\text{and}\quad M_2\cap\big(\delta^{>a}(b)\cup \delta^{>b}(a)\big)=\varnothing 
$$
 for some $a\in\mathcal{A}$ and $b\in\mathcal{B}$. Then, $M_1$ and $M_2$ do not define an edge of the polytope~$P(G)$.
\end{corollary}


\section{Linear Description}
Let us define $Q(G)\subseteq\R^E$ be the polytope described  by the following linear constraints
\begin{align}
x(\delta(v)) \leq 1\quad \forall v \in V\qquad \text{and} \qquad x_e \geq 0\quad \forall e \in E\,,\label{eq:lin_descr_match}\\
x(\delta^{>a}(b))+ x(\delta^{>b}(a)) + x_{\{a,b\}} \geq 1 \quad \forall e:=\{a,b\} \in E \label{eq:lin_descr_stab}
\end{align}
where $x(J) = \sum_{e \in J} x_e$ for any $J \subseteq E$.

Clearly, $P(G)\subseteq Q(G)$ because for every stable matching $M$ in $G$ the point $x:=\chi(M)$  satisfies~\eqref{eq:lin_descr_match} and by~\eqref{eq:stability_def} the point $x$ also satisfies~\eqref{eq:lin_descr_stab}. On the other hand, every integral point in $Q(G)$ equals $\chi(M)$ for some stable matching $M$ in~$G$. In the remaining part of the paper we show that every vertex of $Q(G)$ is integral, thus proving the main theorem.

\begin{theorem}
	For every graph $G$ the polytope $P(G)$ equals $Q(G)$.
\end{theorem}


\begin{lemma}
	For every graph $G$ every vertex of the polytope $Q(G)$ is integral.
\end{lemma}
\begin{proof}
First, let show that every vertex $x$ of $Q(G)$ has a coordinate equal to $0$ or a coordinate equal to $1$. Let us assume the contrary. As every vertex, $x\in \R^E$ is defined as the unique point, which tightly satisfies some $|E|$ constraints describing $Q(G)$. Since $x$ has no zero coordinate, we can assume  that the tight constraints are $x(\delta(v))\le 1$ for $v\in V_x$ and \eqref{eq:lin_descr_stab} for $e\in E_x$, where $|V_x|+|E_x|=|E|$. Moreover, let us assume that we choose the $|E|$ tight constraints so that $|V_x|$ is as large as possible. 

The constraints $x(\delta(v))=1$, $v\in V$ are linearly dependent, in particular $\sum_{a\in\mathcal{A}}\chi(\delta(a))=\sum_{b\in\mathcal{B}}\chi(\delta(b))$. Hence, we have $V_x\subsetneq V$. On the other hand if $a=N_{\max}(b)$ or $a=N_{\min}(b)$ then $e:=\{a,b\}\not\in E_x$. Indeed, $a=N_{\max}(b)$ implies $\delta^{>a}(b)=\varnothing$ then
$$
	1\le x(\delta^{>a}(b))+ x(\delta^{>b}(a)) + x_{\{a,b\}}=x(\delta^{\ge b}(a)) \le x(\delta(a)) \le 1\,,
$$
showing that $\delta^{< b}(a)=\varnothing$ and hence $E_x\setminus e$, $V_x\cup \{a\}$ also define the vertex $x$. The case $a=N_{\min}(b)$ is analogous. Moreover, notice that $N_{\min}(v)\neq N_{\max}(v)$ for $v\in V_x$ since no coordinate of $x$ equals $1$.Thus, 
\begin{multline*}
	|E_x|=\frac{1}{2}\sum_{v\in V} |\delta(v)\cap E_x|\le \frac{1}{2}\sum_{v\in V_x} (|\delta(v)|-2)+\\\frac{1}{2}\sum_{v\in V\setminus V_x} (|\delta(v)|-2)= |E|-|V_x|-\frac{1}{2}|V\setminus V_x|\,,
\end{multline*}
what implies $|E_x|+|V_x|< |E|$, contradiction.

\bigskip

Now let us assume that $G$ is a graph with the minimum number of edges such that $Q(G)$ is not an integral polytope. Let $x$ be a nonintegral vertex of $Q(G)$.

Case $x_{\{a,b\}}=0$ for some $a\in \mathcal{A}$, $b\in\mathcal{B}$ and $e:=\{a,b\}\in E$. In this case, let $x'$ be obtained from $x$ by dropping the coordinate corresponding to $\{a,b\}$, and let $G'$ be the graph with $V(G')=V$ and $E(G')\setminus \{e\}$. Define $H'$ to be the hyperplane $\{x\in \R^{E(G')}: x(\delta^{>a}(b))+ x(\delta^{>b}(a))=1\}$. Then, $x'$ is a vertex of the polytope $P':=P(G')\cap H'$.
But every vertex of $P'$ is either a vertex of $P(G')$ or an intersection of an edge of $P(G')$ with the hyperplane $H'$. Since by Corollary~\ref{cor:edge} there is no edge of $P(G')$ intersecting the hyperplane $H'$, $x'$ is a vertex of $P(G')$. By minimality of $G$, both $x'$ and $x$ are integral, contradiction.

Case $x_{\{a,b\}}=1$ for some $a\in \mathcal{A}$, $b\in\mathcal{B}$. Let $x'$ be obtained by dropping the coordinates corresponding to $\delta(a)\cup\delta(b)$, and let $G'$ be the graph with $V(G')=V\setminus\{a,b\}$ and $E(G')\setminus \big(\delta(a)\cup\delta(b)\big)$. It is straightforward to see that $x'$ is a vertex of $P(G')$. Thus by minimality of $G$, both $x'$ and $x$ are integral, contradiction. 
\end{proof}


\bibliographystyle{alpha}
\bibliography{StableMatchingPolytopeReferences}
\end{document}
